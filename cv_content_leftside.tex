%% Provide the file name containing the sidebar contents as an optional parameter to \cvsection.
%% You can always just use \marginpar{...} if you do
%% not need to align the top of the contents to any
%% \cvsection title in the "main" bar.
\cvsection[cv_content_rightside]{Experience}

    \cvevent{Assistant-engineer}{\href{https://career.huawei.ru/rri/en/}{Huawei RRI}}{july 2022 -- now}{}
    \begin{itemize}
        \item Coauthor of international patent (PCT): Computing performance improving method and
            electronic device (application not published yet).
        \smallskip
        \item Created and took part in creation of architecture-dependent (ARM64) performance/power
            models and algorithms related to Linux kernel scheduler \& frequencies scaling.
    \end{itemize}

% \divider

\cvsection{Pet projects}
    \smallskip

    \begin{itemize}
        \item \href{https://github.com/Vokerlee/riscv64-simulator}
                   {RISC-V 64-bit functional simulator} -- custom simulator with RV64IM instructions
            interpretation via threaded-code, MMU \& TLB.
        \item \href{https://github.com/Vokerlee/EVM}
                   {Echo Virtual Machine (EVM)} -- custom register-based virtual machine
            with incremental garbage collector.
        \item \href{https://github.com/Vokerlee/VSSH}
                   {Vokerlee SSH} -- custom secure shell implementation (TCP + UDP with delivery guarantee)
            via linux virtual terminals \& cgroups.
        \item \href{https://github.com/Vokerlee/linux-api-course/tree/master/5.%20Daemon-backuper}
                   {Incremental $inotify$ daemon-backuper} -- incremental backup-system (deamon), implemented via $inotify$
            Linux kernel subsystem.
        \item Gem5 \& Linux:
        \begin{enumerate}
            \item \href{https://github.com/Vokerlee/gem5-arm-dev}
                       {Gem5} -- added cache PMU events for ARM64, 3-level cache CPU-cluster system.
            \item \href{https://github.com/Vokerlee/linux-6.1-arm-gem5}
                       {Linux 6.1 patches for Gem5} -- implemented $cpufreq$ \& $devfreq$ drivers (+ $clk$)
                for DVFS support for all components in Gem5.
        \end{enumerate}
        \item \href{https://github.com/Vokerlee/llvm-practice}
                   {LLVM practise} -- LLVM front-end of imperative language \& own LLVM back-end of RISC-V like
            architecture with custom graphics extensions.
        \item \href{https://github.com/Vokerlee/cpu-riscv64-pipeline}
                   {RISC-V 64-bit verilog simulator} -- simulator of executable file with RV64I instructions written in
            System Verilog language.

    \end{itemize}

\clearpage

% \cvsection[page2sidebar]{Publications}

\nocite{*}

% \printbibliography[heading=pubtype,title={\printinfo{\faBook}{Books}},type=book]
% \divider
% \printbibliography[heading=pubtype,title={\printinfo{\faFileTextO}{Journal Articles}}, type=article]
% \divider
% \printbibliography[heading=pubtype,title={\printinfo{\faGroup}{Conference Proceedings}},type=inproceedings]

% %% If the NEXT page doesn't start with a \cvsection but you'd
% %% still like to add a sidebar, then use this command on THIS
% %% page to add it. The optional argument lets you pull up the
% %% sidebar a bit so that it looks aligned with the top of the
% %% main column.
% % \addnextpagesidebar[-1ex]{page3sidebar}